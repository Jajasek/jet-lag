\documentclass{book}
\usepackage[czech]{babel}
\usepackage{hyperref}

\title{Pravidla her Jet Lag}
\author{Jáchym Mierva}
\date{2025}

\newenvironment{reasoning}{\begin{small}\itshape}{\end{small}}

\def\timehiding{45 minut}
\def\hidingzoneradius{200 metrů}

\begin{document}

\maketitle
\tableofcontents

\chapter{Hide And Seek: Praha}

\section{Příprava hry}

\subsection{Definice herního plánu}\label{herní plán}

\textbf{Herní plán} tvoří právě všechny dopravní prostředky Pražské MHD poháněné výhradně elektřinou a operující v tarifním pásmu P. Konkrétně jimi jsou
\begin{itemize}
	\item metro,
	\item tramvaj,
	\item trolejbus,
	\item lanovka a
	\item vlak v úseku, na kterém je platný roční kupón PID pro pásmo P.
\end{itemize}
V průběhu hry nesmí hráči používat žádné jiné dopravní prostředky.

Hráči začínají hru u vstupu do stanice metra Florenc.

\begin{reasoning}
	Určitě chceme začínat ve velkém dopravním uzlu, tj. v jedné z přestupních stanic metra. Z nich jsme vybrali Florenc, protože metro A všichni docela známe a bude zajímavé mít možnosti jet někam, kde to tolik neznáme. Navíc prochází Florencí mnoho tramvajových linek a zároveň je poblíž Masarykova nádraží.
\end{reasoning}

\subsection{Rozdělení rolí}

Je určen jeden z hráčů, který bude \textbf{skrývačem}. Ostatní hráči zaujmou roli \textbf{hledačů}. Při větším počtu hráčů se hledači mohou rozdělit do \textbf{týmů} (po dvou až třech osobách), pro modifikaci pravidel v takovém případě viz sekci \ref(týmy).

Skrývač obdrží \textbf{balíček karet} (viz sekci \ref{karty}), každý tým hledačů má mapu herního plánu.

\section{První fáze: skrývání}

Skrývač má \textbf{\timehiding} na to, aby se přesunul na pozici dle libosti. Hledači během této doby nesmí získat žádnou informaci o jeho poloze. V okamžiku \timehiding po začátku skrývání je deklarována \textbf{zóna skrývání} jako vertikální válec o poloměru podstavy \textbf{\hidingzoneradius}, jehož osa prochází aktuální polohou skrývače. Tato zóna \textbf{musí} obsahovat alespoň jednu zastávku či stanici některého z povolených dopravních prostředků. (Přesněji, musí obsahovat netriviální část alespoň jednoho nástupiště.)

Od této chvíle nemá skrývač povoleno opustit svou zónu skrývání.

\section{Druhá fáze: hledání}

Okamžitě po skončení fáze skrývání je spuštěn časovač. Hledači vyráží ze stejné počáteční pozice jako skrývač a jejich cílem je co nejdříve skrývače fyzicky najít, přičemž po celou dobu má skrývač informaci o jejich aktuální poloze (v přesnosti, kterou technologie dovolí). Informace o jeho poloze mohou získávat kladením \textbf{otázek} z množiny uvedené v sekci \ref{otázky}. Skrývač musí na každou otázku odpovědět, a to \textbf{pravdivě}.

\begin{reasoning}
	Pohráváme si i s myšlenkou, že by skrývač mohl čas od času lhát. Konkrétně z každých tří (nebo čtyř, bude třeba vyzkoušet) po sobě jdoucích odpovědí by mohla být nejvýše jedna nepravdivá. Více týmů by se uvažovalo odděleně. Za nepravdivou odpověď by skrývač nedostal odměnu.
\end{reasoning}

Hledači však za každou odpověď, kterou dostanou, zaplatí jistou cenu: dají tím skrývači možnost táhnout karty z balíčku. Počet karet, které táhne, závisí na typu otázky (viz sekci \ref{otázky}). Odpovídá-li skrývač na některou otázku opakovaně, může táhnutí karet provést \textbf{dvakrát}.

Skrývač může karty zahrát kdykoliv, kdy to podmínky na nich napsané dovolí, a to i více karet zároveň. Zahrané karty se odhazují do \textbf{odhazovacího balíčku}. V \textbf{ruce} však může mít \textbf{maximálně 6 karet}. Pokud nastane situace, kdy si do ruky může vzít více karet, než je její kapacita, musí přebývající karty (z ruky či nově získané) okamžitě zahrát či odhodit.

Mezi položením otázky a obdržením odpovědi \textbf{nesmí} hledači klást další otázky.

Vstoupí-li kdykoliv během fáze hledání hledači do zóny skrývání, skrývač se \textbf{nesmí pohybovat}. Může se pak stát, že na některé otázky nedokáže odpovědět. Tuto informaci musí sdělit namísto odpovědi. Přesto si táhne karty. Odejdou-li hledači ze zóny skrývání, může se skrývač opět začít pohybovat.

\section{Množina otázek}\label{otázky}

\section{Balíček karet}\label{karty}

\section{Modifikace pro hru ve více týmech}\label{týmy}

Sdílí množinu otázek, ale navzájem neznají odpovědi. Mohou klást otázky paralelně.

Větší ruka.

\end{document}
