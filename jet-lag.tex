\documentclass{book}
\usepackage[czech]{babel}
\usepackage{hyperref}
\usepackage{enumitem}
\usepackage{mathtools}
\usepackage{wasysym}
\usepackage{xifthen}
\usepackage{pifont}


\newcommand{\ifequals}[3]{\ifthenelse{\equal{#1}{#2}}{#3}{}}
\newcommand{\case}[2]{#1 #2} % Dummy, so \renewcommand has something to overwrite...
\newenvironment{switch}[1]{\renewcommand{\case}{\ifequals{#1}}}{}


\newenvironment{reasoning}{\begin{small}\itshape}{\end{small}}
\newlist{cards}{itemize}{1}
\setlist[cards]{label=\ataribox}
\setlist[itemize]{nosep}

\NewDocumentCommand{\card}{somo}{%
	\edef\tmp{\IfValueT{#2}{[\unexpanded{#2}]}}%
	\expandafter\item\tmp%
	\textbf{%
		#3 %
		(\IfValueTF{#4}{#4}{?}x)%
		\IfBooleanF{#1}{:}%
	}%
}
\newcommand{\cost}{\textbf{Cena za seslání: }}

\def\cross{\ding{55}}

\def\timehiding/{45 minut}  % 3.5h
\def\timehidingmove/{15 minut}  % 1 h
\def\timeanswerquestion/{10 minut}  % photo: 15 min
\def\timecursetravelvisit/{5 minut}  % 10 min
\def\timecursetravelpenalty/{15 minut}  % 30 min
\def\timecurselemonpenalty/{30 minut}  % 1 h
\def\timecursedoorduration/{1 hodinu}  % 3 h
\def\timecursedoorreattempt/{10 minut}  % 15 min
\def\timecursegambler/{30 minut}  % 1 h
\def\timecursedotcountdown/{10 minut}  % 15 min
\def\timecursedotfreeze/{20 minut}  % 30 min
\def\timecursehangmancooldown/{5 minut}  % 10 min
\def\timecursetumblehit/{10 minut}  % 15 min
\def\timecurseturnright/{30 minut}  % 1 h
\def\timecursecensustaker/{15 minut}  % 30 min
\def\timecursebirdguide/{15 minut}  % 15 min
\def\timecursefellowshipcost/{15 minut}

\def\dist#1mi{%
	\begin{switch}{#1}%
		\case{.25}{100 metrů}%
		\case{.5}{200 metrů}%
		\case{1}{350 metrů}%
		\case{3}{750 metrů}%
		\case{5}{1 kilometr}%
		\case{10}{1.5 kilometru}%
		\case{15}{2 kilometry}%
		\case{25}{2 kilometry}%
		\case{50}{3 kilometry}%
		\case{100}{4 kilometry}%
	\end{switch}%
}



\title{Pravidla her Jet Lag}
\author{Žaneta Lipertová, Jáchym Mierva}
\date{2025}


\begin{document}

\maketitle
\tableofcontents

\chapter{Hide And Seek: Praha}

\section{Příprava hry}

\subsection{Definice herního plánu}\label{herní plán}

\textbf{Herní plán} tvoří právě všechny dopravní prostředky Pražské MHD poháněné výhradně elektřinou a operující v tarifním pásmu P, 0 a B, a linky přívozu. Konkrétně jimi jsou
\begin{itemize}
	\item metro,
	\item tramvaj,
	\item trolejbus,
	\item lanovka,
	\item vlak v pásmech P, 0 a B, a
	\item přívoz.
\end{itemize}
Dále do herního plánu spadají linky náhradní autobusové dopravy nahrazující některý z výše uvedených dopravních prostředků.

V průběhu hry nesmí hráči používat žádné jiné dopravní prostředky. Zároveň se musí pohybovat pouze na \textbf{veřejných místech}, jež jsou a budou volně a zdarma přístupná po celou dobu daného kola. (Toto pravidlo mohou hledači porušit, vyžaduje-li to výslovně kletba, viz sekci \ref{kletby})

Hráči začínají hru na křižovatce ulic Na Florenci a Křižíkova (50.0894986N, 14.4372300E).

\begin{reasoning}
	Určitě chceme začínat ve velkém dopravním uzlu, tj. u jedné z přestupních stanic metra. Z nich jsme vybrali Florenc, protože metro A všichni docela známe a je docela krátké. Bude zajímavé mít možnosti jet někam daleko, kde to tolik neznáme.
	
	Bude ale lepší nezačínat hned u vchodu do stanice metra. Pak by totiž hledači měli okamžitě informaci o tom, jakým dopravním prostředkem skrývač pojede. Umístili jsme tedy počáteční bod do poloviny cesty mezi Florencí a Masarykovým nádražím.
\end{reasoning}

\subsection{Určení hrací doby}

Hráči se mohou dohodnout na horním limitu hrací doby. Neskončí-li hra do daného okamžiku, hráči si poznamenají aktuální stav hry (tj. pozice, zónu skrývání, karty v ruce skrývače a aktivní kletby s jejich časovači) a drží jej v tajnosti. Hru pak mohou dohrát v jiný vhodný čas.

\subsection{Rozdělení rolí}

Hra probíhá v několika kolech, jejichž počet hráči určí na začátku hry (a to absolutně či na základě časového limitu). Počet kol by měl být větší či roven počtu hráčů.

Pro každé kolo je určen jeden z hráčů, který bude \textbf{skrývačem}. Ostatní hráči zaujmou roli \textbf{hledačů}. Při větším počtu hráčů se hledači mohou rozdělit do \textbf{týmů} (po dvou až třech osobách), pro modifikaci pravidel v takovém případě viz sekci \ref{týmy}. Rozdělení hráčů by mělo být předem určeno pro všechna kola.

Skrývač prvního kola obdrží \textbf{balíček karet} (viz sekci \ref{karty}), každý tým hledačů má mapu herního plánu.

Během hry by se tým hledačů neměl rozdělit, je považován za jedinou entitu.

\section{První fáze kola: skrývání}\label{skrývání}

Skrývač má \textbf{\timehiding/} na to, aby se přesunul na pozici dle libosti. Hledači během této doby nesmí získat žádnou informaci o jeho poloze. V okamžiku \timehiding/ po začátku skrývání je deklarována \textbf{zóna skrývání} jako množina všech bodů, které jsou vzdáleny nejvýše \textbf{\dist.5mi} od polopřímky počínající v těžišti Země a procházející aktuální polohou skrývače. Tato zóna \textbf{musí být obsluhována} alespoň jedním z povolených dopravních prostředků. Tímto je myšleno:
\begin{itemize}
	\item zóna skrývání musí obsahovat alespoň jeden vchod do některé stanice metra, nebo
	\item se musí dotýkat alespoň jednoho nástupiště, respektive mola povoleného dopravního prostředku povrchové dopravy.
\end{itemize}
Od této chvíle nemá skrývač povoleno opustit svou zónu skrývání.

\section{Druhá fáze kola: hledání}

Okamžitě po skončení fáze skrývání je spuštěn časovač. Hledači vyráží ze stejné počáteční pozice jako skrývač a jejich cílem je co nejdříve skrývače fyzicky najít, přičemž ten má po celou dobu informaci o jejich aktuální poloze (v přesnosti, kterou technologie dovolí). Informace o jeho poloze mohou získávat kladením \textbf{otázek} z množiny uvedené v sekci \ref{otázky}. Skrývač musí na každou otázku odpovědět do \textbf{\timeanswerquestion/}, a to \textbf{pravdivě}.

\begin{reasoning}
	Pohráváme si i s myšlenkou, že by skrývač mohl čas od času lhát. Konkrétně z každých tří (nebo čtyř, bude třeba vyzkoušet) po sobě jdoucích odpovědí by mohla být nejvýše jedna nepravdivá. Více týmů by se uvažovalo odděleně. Za nepravdivou odpověď by skrývač nedostal odměnu.
\end{reasoning}

Hledači však za každou odpověď, kterou dostanou, zaplatí jistou cenu: dají tím skrývači možnost táhnout karty z balíčku. Počet karet, které táhne, závisí na typu otázky (viz sekci \ref{otázky}). Odpovídá-li skrývač na některou otázku opakovaně, může táhnutí karet provést \textbf{dvakrát}.

Skrývač může karty zahrát kdykoliv, kdy to podmínky na nich napsané dovolí, a to i více karet zároveň. Zahrané karty se odhazují do \textbf{odhazovacího balíčku}. V \textbf{ruce} však může mít \textbf{maximálně 6 karet}. Pokud nastane situace, kdy si do ruky může vzít více karet, než je její kapacita, musí přebývající karty (z ruky či nově získané) okamžitě zahrát či odhodit.

Mezi položením otázky a obdržením odpovědi \textbf{nesmí} hledači klást další otázky.

Vstoupí-li kdykoliv během fáze hledání hledači do zóny skrývání, skrývač se \textbf{nesmí pohybovat}. Může se pak stát, že na některé otázky nedokáže odpovědět - tuto informaci musí sdělit namísto odpovědi. I přesto si táhne karty. Odejdou-li hledači ze zóny skrývání, může se skrývač opět začít pohybovat.

\section{Konec kola}

Kolo končí ve chvíli, kdy se hledači dotknou skrývače. Hráči si poznamenají celkový čas hledání společně s časovými bonusy, které skrývač v průběhu kola obdržel díky kartám.

Není-li toto poslední kolo hry, okamžitě začíná fáze skrývání dalšího kola. Dosavadní skrývač předá balíček karet novému skrývači a ten začíná svou fázi skrývání na aktuální pozici všech.

Naopak je-li toto kolo poslední, hra končí. Vítězem je hráč, který v roli skrývače dosáhl nejdelšího času ze všech kol.

\section{Množina otázek}\label{otázky}

%%%%%%%%%%%%%%%%%%%%%%%%%%%%%%
%%%  Hide and Seek: Praha  %%%
%%%  Množina otázek        %%%
%%%%%%%%%%%%%%%%%%%%%%%%%%%%%%

Kdykoliv otázka referuje na aktuální pozici skrývače, je tím myšlena jeho poloha v tom okamžiku, ve kterém se skrývač \textbf{dozví} o položení dané otázky. Skrývač tedy \textbf{nemůže svou aktuální pozici přizpůsobit} právě položené otázce.

\begin{reasoning}
	Oficiální verze hry je adaptovaná na celé Japonsko. Musíme tedy zmenšit používané vzdálenosti. Protože však chceme zmenšit oblast end-game docela málo, zatímco celkovou velikost jsme zmenšili hodně, zdá se přirozené namísto lineárního škálování použít logaritmické.

	Původní hra používá vzdálenosti $0.25mi$ až $100mi$, přičemž poloměr zóny skrývání je $0.5mi$. Použijme pro přepočtení vzdáleností funkci
	\begin{equation*}
		f_\alpha(x) = e^{\alpha\log(1+x)} - 1,\quad x \in [0, \infty)
	\end{equation*}
	pro nějaký faktor $\alpha > 0$. Všimněme si, že pro $\alpha$ splňující $f_\alpha(100mi) = 4km$ (kde $4km$ se zdá býti rozumný poloměr pro největší radar) dostáváme $f_\alpha(0.5mi) \simeq 210m$. To docela dobře odpovídá námi zvolenému poloměru zóny skrývání. Budu tedy používat tuto funkci, se zaokrouhlováním na rozumně použitelné vzdálenosti. (Pozn. dovolím si 1.77km i 2.24km zaokrouhlit na 2km.)

	Existuje však ještě jiný možný myšlenkový pochod. Protože hrajeme na menší ploše, dala by se nad naše hra připodobnit nějakému kratšímu intervalu původní hry, od jistého bodu do konce. V tomto kratším intervalu už jsou rozumně použitelné jen některé otázky z původní hry, konkrétně ty s menšími vzdálenostmi. Dávalo by tedy smysl namísto škálování prostě jen vypustit tu část otázek, jejichž měřítko neodpovídá našemu kontextu. To budiž předmětem debaty.

	Případně je možné zkombinovat oba přístupy.
\end{reasoning}

\subsection{Porovnávání}

\textbf{Cena:} Skrývač táhne 3 karty, z nichž si jednu vybere a zbytek odhodí.

\textbf{Prototyp otázky:} Je, ve tvé aktuální pozici, tvůj \verb|<objekt>| stejný jako náš?

Zde \verb|<objekt>| může být vybírán z následujícího seznamu:

\begin{itemize}
	\item nejbližší letiště (Uvažujeme 4 Pražská letiště: Václava Havla, Kbely, Letňany a Točná. Vzdálenost se počítá od nejbližšího bodu ranveje.)
	\item Městská část
	\item Nejbližší metro
	\item První písmeno ulice
	\item Nejbližší velvyslanectví
	\item Břeh Vltavy
	\item Nejbližší budova MFF UK
\end{itemize}

Dále tato kategorie obsahuje následující otázky:

\begin{itemize}
	\item Obsluhuje linka, jíž právě cestuji, zónu skrývání (dle definice v sekci \ref{skrývání})?
	\item Je tvá aktuální nadmořská výška vyšší než moje?
\end{itemize}


\subsection{Měření}

\textbf{Cena:} Skrývač táhne 3 karty, z nichž si jednu vybere a zbytek odhodí.

\textbf{Prototyp otázky:} Jsi, na své aktuální pozici, ke tvému nejbližšímu \verb|<objektu>| blíže než my k našemu nejbližšímu?

Zde \verb|<objekt>| může být vybírán z následujícího seznamu:

\begin{itemize}
	\item Letiště
	\item Vltava
	\item McDonald's
	\item Metro
	\item Dálnice
	\item Linka povoleného dopravního prostředku
	\item budova MFF UK
\end{itemize}

\subsection{Teploměr}

\textbf{Cena:} Skrývač táhne 2 karty, z nichž si jednu vybere a druhou odhodí.

Hledači oznámí skrývači, že jejich aktuální pozice je \textbf{začátkem teploměru} o délce zvolené ze seznamu níže. Poté na libovolné pozici, která \textbf{není blíže} počátku teploměru než je zvolená délka, mohou teploměr \textbf{ukončit}. Skrývač následně odpovídá na otázku: "Jsi v tuto chvíli blíže konci teploměru, než jeho začátku?"

Možné vzdálenosti jsou:
\begin{itemize}
	\item \dist.5mi,
	\item \dist5mi,
	\item \dist15mi,
	\item \dist50mi.
\end{itemize}

\subsection{Radar}

\textbf{Cena:} Skrývač táhne 2 karty, z nichž si jednu vybere a druhou odhodí.

\textbf{Prototyp otázky:} Je tvá aktuální vzdálenost k nám nejvýše \verb|<vzdálenost>|?

Zde \verb|<vzdálenost>| může být vybírána z následujícího seznamu:
\begin{itemize}
	\item \dist.25mi,
	\item \dist.5mi,
	\item \dist1mi,
	\item \dist3mi,
	\item \dist5mi,
	\item \dist10mi,
	\item \dist25mi,
	\item \dist50mi,
	\item \dist100mi,
	\item vzdálenost libovolně zvolená hledači.
\end{itemize}

\subsection{Chapadla}

\textbf{Cena:} Skrývač táhne 4 karty, z nichž si 2 vybere a zbytek odhodí.

Hledači určí \textbf{poloměr} kruhu se středem v jejich aktuální pozici a \textbf{třídu objektů} v tomto kruhu, a to z následujícího seznamu možností. Pokud je skrývač v jimi určeném kruhu, sdělí jim, \textbf{který} z těchto objektů je mu nejbližší. Není-li v daném kruhu, tuto skutečnost hledačům \textbf{sdělí namísto odpovědi}.

\begin{itemize}
	\item Pro kruh o poloměru \dist1mi lze volit třídy objektů:
	\begin{itemize}
		\item zastupitelské úřady (\href{https://cs.wikipedia.org/wiki/Seznam_zastupitelsk\%C3\%BDch_\%C3\%BA\%C5\%99ad\%C5\%AF_v_\%C4\%8Cesk\%C3\%A9_republice}{seznam lze nalézt zde}),
		\item zastávky/stanice veškeré Pražské MHD (vzdálenost se počítá od nejbližšího bodu nástupiště povrchové dopravy či od nejbližšího vchodu do stanice metra),
		\item obchody s potravinami,
		\item lékárny.
	\end{itemize}
	\item Pro kruh o poloměru \dist15mi lze volit třídy objektů:
	\begin{itemize}
		\item fakulty UK (vzdálenost se počítá od nejbližší budovy dané fakulty),
		\item smyčky tramvají (\href{https://mapy.com/s/fobosuzuzu}{mapa všech smyček je zde}),
		\item mosty a lávky přes Vltavu,
		\item pobočky Billa.
	\end{itemize}
\end{itemize}

\subsection{Fotografie}

\textbf{Cena:} Skrývač táhne 4 karty, z nichž si 2 vybere a zbytek odhodí.

Skrývač pošle hledačům jednu fotografii, která obsahuje daný objekt z možností níže. Hledači nemají dovoleno pro lokalizaci fotografií cíleně používat fotky dostupné v mapách, na internetu apod. Mohou používat vlastní galerii.

\begin{reasoning}
	Přijde nám, že fotky nám prozradí obecně mnohem více informací než hráčům původní hry, neboť Praha pro nás zdaleka není neznámým místem. Navrhujeme je tedy zdražit na úroveň chapadel, abychom předešli Samovu \uv{Photos are cheap!}.
\end{reasoning}

Fotografie by měla být nějak rozumně nepřiblížená, podobně jako v originální sérii Hide And Seek: Japan. Fotka budovy musí obsahovat obě strany průmětu budovy a střechu.

\begin{itemize}
	\item Bod země vypadající nejvýše ze schovávačovy aktuálně nejbližší zastávky povoleného dopravního prostředku (tj. z libovolného nástupiště či mola povrchové dopravy, nebo z libovolného vchodu do stanice metra). Tento bod je nutné vyfotit z oné zastávky.
	\item Nejbližší McDonald's, existuje-li takový v zóně skrývání. Fotka musí obsahovat alespoň jeden vchod a alespoň jeden poutač.
	\item Nejširší ulice v zóně skrývání. Fotka musí obsahovat obě strany ulice.
	\item Lidmi vytvořená struktura vypadající nejvyšší ze schovávačovy aktuálně nejbližší zastávky povoleného dopravního prostředku. Tuto budovu je však možno vyfotit odkudkoli ze zóny skrývání.
	\item Nejbližší obchod s potravinami. Fotka musí obsahovat alespoň jeden vchod a alespoň jeden poutač.
	\item Nejbližší nástupiště povoleného dopravního prostředku. Fotka musí obsahovat alespoň polovinu plochy nástupiště.
	\item Lidmi vytvořená struktura vypadající nejvyšší ze schovávačovy současné pozice. Tuto strukturu je však možno vyfotit odkudkoli ze zóny skrývání.
	\item 5 budov. Namísto běžného pravidla musí fotka obsahovat základnu a alespoň jedno patro každé z nich.
	\item Selfie (tj. fotka obsahující celý obličej schovávače) ze schovávačovy současné pozice.
	\item Obrys procházky o délce alespoň \dist.5mi, obsahující alespoň 4 zatáčky (?, původně 6).
	\item Libovolný strom, je-li nějaký na dohled. Fotka musí být focena ze schovávačovy současné pozice a musí obsahovat celou korunu stromu.
	\item Pohled přímo vzhůru. Fotka musí být focena ze schovávačovy současné pozice.
	\item Tvar ulice, která je nejbližší schovávačově současné pozici.
\end{itemize}



\section{Balíček karet}\label{karty}

%%%%%%%%%%%%%%%%%%%%%%%%%%%%%%
%%%  Hide and Seek: Praha  %%%
%%%  Balíček karet         %%%
%%%%%%%%%%%%%%%%%%%%%%%%%%%%%%


\subsection{Časové bonusy}

Tyto karty nelze zahrát. Po skončení kola se k celkovému času přičte hodnota všech časových bonusů, které v tu chvíli drží skrývač v ruce.

\begin{cards}
	\card*[2\pikyb\pikyr, 3-5\srdceb\srdcer\karyb\karyr\pikyb\pikyr\krizeb\krizer]{Bonus 5 minut}[26]
	\card*[6-7\srdceb\srdcer\karyb\karyr\pikyb\pikyr\krizeb\krizer]{Bonus 10 minut}[16]
	\card*[8\srdceb\srdcer\karyb\karyr\pikyb\pikyr\krizeb\krizer]{Bonus 15 minut}[8]
	\card*[A\karyb\karyr]{Bonus 20 minut}[2]
\end{cards}

\subsection{Akční karty}

\begin{cards}
	\card[2\srdceb\srdcer\karyb\karyr]{Veto}[4] Vyber otázku, která ještě nebyla položena či byla právě nyní poprvé položena. Odteď se považuje za již položenou. Pokud jsi tuto kartu zahrál v reakci na položení této otázky, nemusíš na ni odpovědět (přičemž zřejmě nemáš právo na příslušnou odměnu).

	\card*[2\krizeb, 9\srdceb\karyb\pikyb\krizeb]{Odhoď 1, dober si 2}[5]

	\card*[2\krizer, 9\srdcer\karyr\pikyr\krizer]{Odhoď 2, dober si 3}[5]

	\card[10\srdceb\karyb\pikyb\krizeb\srdcer]{Náhodná otázka}[5] Zahraj okamžitě poté, co je položena otázka. Neodpovídej na položenou otázku - namísto toho náhodně zvol \textbf{jinou} dosud nepoloženou otázku ze stejné kategorie. Pokud taková neexistuje, kartu nelze použít. Sděl hledačům zvolenou otázku a odpověz na ni.

	Jak původně položená otázka, tak náhodně zvolená, jsou považovány za \textbf{položené}. Karty si však táhneš pouze jednou.

	\card[10\karyr\pikyr\krizer]{Duplikát}[3] V okamžiku, kdy hraješ tuto kartu, zvol jinou kartu ze své ruky. Tato karta má efekt zvolené karty.

	Máš-li tuto kartu v ruce na konci kola, můžeš zduplikovat libovolný časový bonus.

	\card*[A\srdceb]{Lízni kartu a zvyš kapacitu ruky o 1}[1]
	
	\card[A\srdcer]{Přesun}[1] Zahraj, pokud hledači \textbf{nejsou v zóně skrývání}. Sděl jim pozici středu zóny skrývání a odhoď všechny karty z ruky. Máš nyní \textbf{\timehidingmove/} na to, aby ses libovolně přesunul a deklaroval \textbf{novou zónu skrývání}. Během této doby se hledači nesmí pohnout ani klást otázky. Tento čas se však \textbf{nepočítá} do celkového času hledání.
	
	Pokud jsou hledači v dopravním prostředku, musí jej při nejbližší příležitosti opustit.
\end{cards}

\subsection{Kletby}\label{kletby}

\begin{cards}
	
	\card[J\srdceb]{Kletba exotické kuchyně}[1] Zahraj ve chvíli, kdy jsi uvnitř restaurace podávající explicitně kuchyni z jistého cizího státu. Hledači pak nesmí položit \textbf{žádnou otázku} do chvíle, kdy najdou restauraci podávající kuchyni ze státu, jehož vzdálenost od České republiky je větší či rovna.
	
	\card[Q\srdceb]{Kletba sčítače lidu}[1] Hraj, pokud hledači právě \textbf{necestují} dopravním prostředkem. Hledači musí okamžitě vyrazit nejrychlejší cestou k \textbf{radnici} té městské části, v níž se aktuálně nachází. Jakmile tam dorazí, musí \textbf{odhadnout počet obyvatel} této městské části, s přesností na \textbf{25\%}. Pokud je jejich odhad špatný, obdržíš bonus \timecursecensustaker/.
	
	\cost následující otázka hledačů je zdarma.
	
	\card[K\srdceb]{Kletba šíleného matematika}[1] Hledači musí nalézt desítkový zápis \textbf{každého prvočísla menšího než 24}. Tato pročísla se musí vyskytovat v okolním světě, nikoliv na objektech ve vlastnictví hledačů. Počítá se pouze \textbf{samostatný zápis}, nikoliv jako součást zápisu většího čísla. Dokud není kletba splněna, nesmí hledači položit \textbf{žádnou otázku}.
	
	\cost před sesláním této kletby musíš nalézt desítkový zápis nějakého prvočísla mezi 24 a 100 a sdělit toto prvočíslo hledačům.
	
	\card[J\karyb]{Kletba zmrzlé tečky}[1] Hraj, pokud jsou od tebe hledači vzdáleni nejméně \dist10mi. Urči bod vzdálený alespoň \dist.5mi od aktuální pozice hledačů. Pokud přesně za \timecursedotcountdown/ budou ve vzdálenosti nejvýše \dist.25mi od daného bodu, nesmí se po následujících \timecursedotfreeze/ pohnout.
	
	\card[Q\karyb]{Kletba nekonečného kutálení}[1] Hoď \textbf{1d12}. Hledači musí hodit na \textbf{1d12} stejné nebo vyšší číslo, přičemž tato kostka se musí kutálet \textbf{alespoň 30 metrů}. Dokud se jim to nepodaří, nesmí položit \textbf{žádnou otázku}.
	
	Kostka hledačů se musí kutálet celou dobu \textbf{bez jakékoli pomoci}, s využitím pouze energie hodu a gravitační síly. Pokud hledači někoho touto kostkou trefí, je ti udělen bonus \timecursetumblehit/.
	
	\card[K\karyb]{Kletba vypadávající paměti}[1] Po zbytek kola bude v každém okamžiku jeden typ otázek \textbf{zablokovaný}. Určí se náhodně hodem 1d5, přičemž tento hod je opakován po každé položené otázce.
	
	Pro účely této kletby se použití karty Veto nepovažuje za položení otázky.
	
	\cost odhoď časový bonus z ruky.
	
	\card[J\pikyb]{Kletba městského průzkumníka}[1] Do konce kola platí, že hledači \textbf{nesmí klást otázky} ve chvíli, kdy jsou v \textbf{dopravním prostředku} či na \textbf{zastávce}/\textbf{stanici}.
	
	\cost odhoď 2 karty z ruky.
	
	\card[Q\pikyb]{Kletba mohyly}[1] Od chvíle, kdy tuto kartu zahraješ, máš jeden pokus na to, abys postavil mohylu lineárně uspořádaných kamenů. Každý z těchto kamenů se smí dotýkat pouze kamenů \textbf{s ním sousedících} - vyjma prvního, který stojí na zemi. Jakmile tvoje věž spadne, je definována její \textbf{výška} jako nejvyšší počet kamenů takový, že věž s tímto počtem kamenů byla stabilní (tj. stála nehnutě alespoň 5 vteřin). Hledači následně musí stejným způsobem postavit mohylu \textbf{o stejné nebo větší výšce}. Dokud není kletba splněna, hledači nesmí položit \textbf{žádnou otázku}.
	
	\card[K\pikyb]{Kletba nohou hazardního hráče}[1] Během následujících \timecursegambler/, kdykoliv chtějí hledači jít jakýmkoliv směrem, musí hodit 1d8. Následně mohou udělat pouze tolik kroků, kolik padlo. Pak mohou házet znovu.
	
	\cost Musíš uspět hod \textbf{1d20, DC 11}, jinak nemá tato kletba žádný efekt.
	
	\card[A\pikyb]{Kletba černé díry}[1] Urči poloměr kruhu se středem v aktuální pozici hledačů. Tento poloměr musí být menší či roven \textbf{třem čtvrtinám jejich vzdálenosti od tebe}. Po zbytek kola platí, že kdykoliv jsou hledači uvnitř určeného kruhu, \textbf{nesmí pokládat otázky}.
	
	\cost odhoď kletbu z ruky.
	
	\begin{reasoning}
		V původní hře byl poloměr pevně daný hodnotou 75mi, s podmínkou vzdálenosti alespoň 100mi mezi hledači a zónou skrývání. Přepočtená vzdálenost \dist100mi je však příliš velká a nemohl jsem se rozhodnout, jaká by byla adekvátní. Tak mě napadlo to udělat variabilní - čím silnější efekt skrývač použije, tím více informací o své poloze prozradí. Za tuto variabilitu však musí zaplatit další kletbou.
	\end{reasoning}
	
	\card[J\krizeb]{Kletba obratu}[1] Zahraj, pokud hledači právě cestují dopravním prostředkem, jehož následující stanice je od tebe \textbf{vzdálenější} než předešlá. Hledači musí na následující stanici \textbf{vystoupit} a poté, je-li to možné, se vrátit \textbf{stejnou linkou v opačném směru} o alespoň jednu zastávku.
	
	\card[Q\krizeb]{Kletba všímavého spotřebitele}[1] Hledači musí zakoupit produkt či získat vstup na lokaci, na niž viděli reklamu. Tato reklama musí být nalezena v \textbf{reálném světě} (nikoliv na digitálním zařízení hledače) a musí být od samotného produktu či lokace vzdálena alespoň 100m. Dokud není kletba splněna, hledači nesmí položit \textbf{žádnou otázku}.
	
	\cost následující otázka hledačů je zdarma.
	
	\begin{reasoning}
		Protože hrajeme v Praze, je tato kletba spíše lehká. Tak jsem se rozhodl alespoň trochu zvýšit limit vzdálenosti.
	\end{reasoning}
	
	\card[K\krizeb]{Kletba zatáčky vpravo}[1] Během následujících \timecurseturnright/ musí hledači na každé křižovatce (včetně chodeb v budovách apod.) \textbf{odbočit vpravo} (o libovolný úhel $\phi \in (0, \pi)$), je-li to možné. Není-li to možné, musí odbočit \textbf{o co nejmenší úhel doleva}. Pokud se hledači ocitnou ve slepé uličce či v situaci, že by museli jít dále než \dist.5mi bez jediné křižovatky s odbočkou vpravo, mohou se otočit o $\pi$.
	
	\cost odhoď 1 kletbu a 1 další libovolnou kartu z ruky.
	
	\card[A\krizeb]{Kletba vymytého mozku}[1] Vyber tři otázky různých typů. Hledači nesmí do konce kola tyto otázky položit.
	
	\cost odhoď všechny karty z ruky.

	\card[J\srdcer]{Kletba nezkušeného cestovatele}[1] Hraj, pokud hledači právě \textbf{necestují} dopravním prostředkem. Urči libovolnou veřejně dostupnou lokaci vzdálenou nejvýše \dist.5mi od hledačů. Ti tam musí dojít a strávit tam alespoň \timecursetravelvisit/. \textbf{Až poté mohou položit další otázku.}

	Hledači ti musí z dané lokace přinést nějaký objekt jako \textbf{suvenýr}. Ztratí-li ho předtím, než ti ho předají, je ti udělen časový bonus \timecursetravelpenalty/.

	\cost cílová lokace od tebe musí být dál než aktuální pozice hledačů.
	
	\card[Q\srdcer]{Kletba zoologa}[1] Vyfoť divokou rybu, ptáka, savce, plaze, obojživelníka či hmyz. Hledači musí vyfotit divoké zvíře stejné kategorie. Do té doby nesmí položit \textbf{žádnou otázku}.
	
	\card[K\srdcer]{Kletba společenstva Dveří}[1] Hraj, pokud hledači \textbf{nejsou v zóně skrývání}. Hledači musí projít \textbf{37 různými dveřmi}, a to takovými, které nelze otevřít či zavřít jinak, než \textbf{ručně}. Dokud není kletba splněna, nesmí hledači položit \textbf{žádnou otázku}.
	
	\cost odhoď z ruky časové bonusy v souhrnné délce alespoň \textbf{\timecursefellowshipcost/}.
	
	\card[J\karyr]{Kletba olympijské vesnice}[1] Hledači musí najít alespoň 5 různých vlajek nějakých geografických oblastí. Nemusí být nutně schopni je identifikovat. Tyto vlajky se musí nacházet v okolním světě, nikoliv na objektech ve vlastnictví hledačů. Dokud není kletba splněna, nesmí hledači položit \textbf{žádnou otázku}.
	
	\cost odhoď jednu kartu z ruky.
	
	\card[Q\karyr]{Kletba ptačího průvodce}[1] Když zahraješ tuto kartu, máš jeden pokus natočit co nejdelší \textbf{nepřetržitý záběr ptáka}. Točíš-li ptáků více zároveň, musíš se předem rozhodnout, \textbf{kterého} z nich sleduješ. Záběr končí, jakmile daný pták není viditelný ani na jediném pixelu, nebo jakmile jeho délka přesáhne \textbf{\timecursebirdguide/}. Hledači poté musí natočit záběr ptáka o stejné nebo větší délce. Dokud není kletba splněna, hledači nesmí položit \textbf{žádnou otázku}.

	\card[K\karyr]{Kletba operátora dopravního podniku}[1] Hraj, pokud hledači \textbf{nejsou v zóně skrývání}. Hledači musí najít zastávku, kde jezdí alespoň tři různé linky povolených dopravních prostředků, a správně tipnout, jaká přijede první. V případě neúspěchu mohou tip libovolně mnohokrát opakovat. Dokud není kletba splněna, nesmí hledači používat jízdní řády v jakékoliv formě.

	\cost dokud držíš tuto kartu v ruce, nesmíš používat jízdní řády v jakékoliv formě.
	
	\card[J\pikyr]{Kletba Honzy Žižky}[1] Hraj, pokud je vzdálenost hledačů od Žižkovské věže nejvýše dvojnásobek jejich vzdálenosti k tobě. Hledači musí vyfotit Žižkovskou věž tak, aby byla vidět všechna její patra. Dokud není kletba splněna, hledači nesmí položit \textbf{žádnou otázku}.
	
	\card[Q\pikyr]{Kletba ukrytého oběšence}[1] Hledači musí porazit skrývače ve hře \textbf{šibenice}, jež začíná okamžitě.
	
	Skrývač zvolí \textbf{libovolné spisovné české slovo} mající alespoň 4 písmena, hledači následně písmena hádají. Po \textbf{sedmi} neúspěšných pokusech prohrávají. Hraje se se standardní českou abecedou vyjma diakritiky, tj. \textbf{28 písmen}.
	
	Skrývač musí na každý dotaz reagovat do 30 vteřin. Po porážce nemůžou hledači v následujících \timecursehangmancooldown/ách skrývače vyzvat k další hře. Po třech porážkách či jedné výhře je kletba zrušena.
	
	Dokud není kletba zrušena, hledači nesmí položit \textbf{žádnou otázku} ani nastoupit do \textbf{žádného dopravního prostředku}.
	
	\cost odhoď 2 karty z ruky.

	\card[K\pikyr]{Kletba návštěvníka z dalekých krajů}[1] Najdi nějakou státní poznávací značku, u níž jsi schopen určit oblast původu. Hledači pak musí najít státní poznávací značku pocházející z oblasti, která je stejně či více vzdálená. Dokud není kletba splněna, nesmí hledači položit \textbf{žádnou otázku}.
	
	\card[A\pikyr]{Kletba zabouchnutých dveří}[1] Po následující \timecursedoorduration/ platí, že kdykoliv chtějí hledači projít dveřmi do/z budovy či dopravního prostředku, musí uspět hod \textbf{1d20, DC 9}. Neuspějí-li, nesmí do/z daného prostoru projít, a to \textbf{ani jinými dveřmi}. Po uplynutí \timecursedoorreattempt/ se mohou pokusit uspět znovu.
	
	\cost odhoď libovolnou kartu z ruky.
	
	\card[J\krizer]{Kletba luxusního auta}[1] Vyfoť libovolné auto. Hledači pak musí vyfotit auto o stejné nebo vyšší pořizovací ceně. Do té doby nesmí položit \textbf{žádnou otázku}.
	
	\card[Q\krizer]{Kletba ztraceného turisty}[1] Pošli hledačům nepřiblížený obrázek ze streetview libovolné mapové služby. Snímek musí být rovnoběžný s horizontem a musí obsahovat alespoň jednu lidmi vytvořenou strukturu jinou než silnici či cestu. Daná lokace musí být vzdálena nejvýše \dist.25mi od současné pozice hledačů.
	
	Hledači musí bez použití digitální technologie dojít na dané místo a poslat ti fotku pro ověření. Dokud není kletba splněna, hledači nesmí položit \textbf{žádnou otázku} ani použít \textbf{dopravní prostředek}.
	
	\cost Hledači se musí nacházet venku.
	
	\card[K\krizer]{Kletba mostního trolla}[1] Zahraj, pokud jsou od tebe hledači vzdáleni nejméně \dist50mi. Hledači musí položit svoji následující otázku ve chvíli, kdy jsou \textbf{pod mostem}.

	\card[A\krizer]{Kletba vybíravého cestovatele}[1]
	\begin{itemize}
		\item najdi tramvaj, hledači po následující hodinu mohou používat jen linky obsahující číslo stejné parity
		\item vyfoť tramvaj a pošli hledačům, po následující hodinu je jediný dopravní prostředek, který mohou hledači používat, tramvaj stejného typu
	\end{itemize}
	
	\card[\cross]{Curse of the ransom note}[0]
	
	\card[\cross]{Kletba citronového amuletu}[0] Každý hledač musí najít citrón a připevnit ho na vnější vrstvu oblečení či kůži. Do té doby nesmí položit žádnou otázku. Pokud v celém zbytku kola nastane situace, že se některý z těchto citrónů nedotýká hledače, je ti udělen časový bonus \timecurselemonpenalty/.
	
	\cost odhoď nějakou akční kartu z ruky.
\end{cards}


\section{Modifikace pro hru ve více týmech}\label{týmy}

Rozhodnou-li se hledači rozdělit do více týmů, hra probíhá stejným způsobem s následujícími změnami.

\subsection{Průběh kola}

Týmy hledačů navzájem neznají svou aktuální pozici.

Kolo skončí tehdy, když alespoň jeden z hledačů skrývače dotkne. V tu chvíli je pozice skrývače oznámena všem týmům, které se na tuto pozici nejrychlejším způsobem přesunou. Až poté začne následující kolo.

Dorazí-li však následující skrývač na danou pozici dříve než některý jiný tým, může se rozhodnout začít svou fázi skrývání před jejich příjezdem, uvědomujíce si, že tím riskuje prozrazení informací o své zóně skrývání. Fáze hledání však musí začít až tehdy, když se všechny týmy dostavily na poslední pozici předchozího skrývače.

\subsection{Otázky}

Týmy sdílí jednu množinu otázek. Jakmile tedy jeden z týmů položí nějakou otázku, sdělí tuto skutečnost všem ostatním týmům. Tato otázka je pak i pro ně považována za položenou. Odpověď však dostane pouze ten tým, který otázku položil.

Lhůta skrývače pro zodpovězení otázky může běžet pro několik týmů paralelně.

\subsection{Karty}

Limit počtu karet v ruce skrývače se zvyšuje na 8.

Sesílá-li skrývač kletbu, musí si vybrat pouze jeden tým, na který ji sešle. Pouze na tento tým se pak vážou veškeré podmínky a efekty seslané kletby. Výjimkou jsou takové kletby, které manipulují s množinou otázek - ty afektují všechny týmy.

Akční karty ze své podstaty afektují buďto všechny týmy, nebo žádné.

\end{document}
