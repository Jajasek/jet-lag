%%%%%%%%%%%%%%%%%%%%%%%%%%%%%%
%%%  Hide and Seek: Praha  %%%
%%%  Balíček karet         %%%
%%%%%%%%%%%%%%%%%%%%%%%%%%%%%%


\subsection{Časové bonusy}

Tyto karty nelze zahrát. Po skončení kola se k celkovému času přičte hodnota všech časových bonusů, které v tu chvíli drží skrývač v ruce.

\begin{cards}
	\card*[2\pikyb\pikyr, 3-5\srdceb\srdcer\karyb\karyr\pikyb\pikyr\krizeb\krizer]{Bonus 5 minut}[26]
	\card*[6-7\srdceb\srdcer\karyb\karyr\pikyb\pikyr\krizeb\krizer]{Bonus 10 minut}[16]
	\card*[8\srdceb\srdcer\karyb\karyr\pikyb\pikyr\krizeb\krizer]{Bonus 15 minut}[8]
	\card*[A\karyb\karyr]{Bonus 20 minut}[2]
\end{cards}

\subsection{Akční karty}

\begin{cards}
	\card[2\srdceb\srdcer\karyb\karyr]{Veto}[4] Vyber otázku, která ještě nebyla položena či byla právě nyní poprvé položena. Odteď se považuje za již položenou. Pokud jsi tuto kartu zahrál v reakci na položení této otázky, nemusíš na ni odpovědět (přičemž zřejmě nemáš právo na příslušnou odměnu).

	\card*[2\krizeb, 9\srdceb\karyb\pikyb\krizeb]{Odhoď 1, dober si 2}[5]

	\card*[2\krizer, 9\srdcer\karyr\pikyr\krizer]{Odhoď 2, dober si 3}[5]

	\card[10\srdceb\karyb\pikyb\krizeb\srdcer]{Náhodná otázka}[5] Zahraj okamžitě poté, co je položena otázka. Neodpovídej na položenou otázku - namísto toho náhodně zvol \textbf{jinou} dosud nepoloženou otázku ze stejné kategorie. Pokud taková neexistuje, kartu nelze použít. Sděl hledačům zvolenou otázku a odpověz na ni.

	Jak původně položená otázka, tak náhodně zvolená, jsou považovány za \textbf{položené}. Karty si však táhneš pouze jednou.

	\card[10\karyr\pikyr\krizer]{Duplikát}[3] V okamžiku, kdy hraješ tuto kartu, zvol jinou kartu ze své ruky. Tato karta má efekt zvolené karty.

	Máš-li tuto kartu v ruce na konci kola, můžeš zduplikovat libovolný časový bonus.

	\card*[A\srdceb]{Lízni kartu a zvyš kapacitu ruky o 1}[1]
	
	\card[A\srdcer]{Přesun}[1] Zahraj, pokud hledači \textbf{nejsou v zóně skrývání}. Sděl jim pozici středu zóny skrývání a odhoď všechny karty z ruky. Máš nyní \textbf{\timehidingmove/} na to, aby ses libovolně přesunul a deklaroval \textbf{novou zónu skrývání}. Během této doby se hledači nesmí pohnout ani klást otázky. Tento čas se však \textbf{nepočítá} do celkového času hledání.
	
	Pokud jsou hledači v dopravním prostředku, musí jej při nejbližší příležitosti opustit.
\end{cards}

\subsection{Kletby}\label{kletby}

\begin{cards}
	
	\card[J\srdceb]{Kletba exotické kuchyně}[1] Zahraj ve chvíli, kdy jsi uvnitř restaurace podávající explicitně kuchyni z jistého cizího státu. Hledači pak nesmí položit \textbf{žádnou otázku} do chvíle, kdy najdou restauraci podávající kuchyni ze státu, jehož vzdálenost od České republiky je větší či rovna.
	
	\card[Q\srdceb]{Kletba sčítače lidu}[1] Hraj, pokud hledači právě \textbf{necestují} dopravním prostředkem. Hledači musí okamžitě vyrazit nejrychlejší cestou k \textbf{radnici} té městské části, v níž se aktuálně nachází. Jakmile tam dorazí, musí \textbf{odhadnout počet obyvatel} této městské části, s přesností na \textbf{25\%}. Pokud je jejich odhad špatný, obdržíš bonus \timecursecensustaker/.
	
	\cost následující otázka hledačů je zdarma.
	
	\card[K\srdceb]{Kletba šíleného matematika}[1] Hledači musí nalézt desítkový zápis \textbf{každého prvočísla menšího než 24}. Tato pročísla se musí vyskytovat v okolním světě, nikoliv na objektech ve vlastnictví hledačů. Počítá se pouze \textbf{samostatný zápis}, nikoliv jako součást zápisu většího čísla. Dokud není kletba splněna, nesmí hledači položit \textbf{žádnou otázku}.
	
	\cost před sesláním této kletby musíš nalézt desítkový zápis nějakého prvočísla mezi 24 a 100 a sdělit toto prvočíslo hledačům.
	
	\begin{reasoning}
		Možná i vyfotit a poslat, ať to hledačům aspoň něco řekne?
	\end{reasoning}
	
	\card[J\karyb]{Kletba zmrzlé tečky}[1] Hraj, pokud jsou od tebe hledači vzdáleni nejméně \dist10mi. Urči bod vzdálený alespoň \dist.5mi od aktuální pozice hledačů. Pokud přesně za \timecursedotcountdown/ budou ve vzdálenosti nejvýše \dist.25mi od daného bodu, nesmí se po následujících \timecursedotfreeze/ pohnout.
	
	\card[Q\karyb]{Kletba nekonečného kutálení}[1] Hoď \textbf{1d12}. Hledači musí hodit na \textbf{1d12} stejné nebo vyšší číslo, přičemž tato kostka se musí kutálet \textbf{alespoň 30 metrů}. Dokud se jim to nepodaří, nesmí položit \textbf{žádnou otázku}.
	
	Kostka hledačů se musí kutálet celou dobu \textbf{bez jakékoli pomoci}, s využitím pouze energie hodu a gravitační síly. Pokud hledači někoho touto kostkou trefí, je ti udělen bonus \timecursetumblehit/.
	Otázka: každý hod musí být 30 metrů?

	\card[K\karyb]{Kletba vypadávající paměti}[1] Po zbytek kola bude v každém okamžiku jeden typ otázek \textbf{zablokovaný}. Určí se náhodně hodem 1d5, přičemž tento hod je opakován po každé položené otázce.
	
	Pro účely této kletby se použití karty Veto nepovažuje za položení otázky.
	
	\cost odhoď časový bonus z ruky.
	
	\card[J\pikyb]{Kletba městského průzkumníka}[1] Do konce kola platí, že hledači \textbf{nesmí klást otázky} ve chvíli, kdy jsou v \textbf{dopravním prostředku} či na \textbf{zastávce}/\textbf{stanici}.
	
	\cost odhoď 2 karty z ruky.
	
	\card[Q\pikyb]{Kletba mohyly}[1] Od chvíle, kdy tuto kartu zahraješ, máš jeden pokus na to, abys postavil mohylu lineárně uspořádaných kamenů. Každý z těchto kamenů se smí dotýkat pouze kamenů \textbf{s ním sousedících} - vyjma prvního, který stojí na zemi. Jakmile tvoje věž spadne, je definována její \textbf{výška} jako nejvyšší počet kamenů takový, že věž s tímto počtem kamenů byla stabilní (tj. stála nehnutě alespoň 5 vteřin). Hledači následně musí stejným způsobem postavit mohylu \textbf{o stejné nebo větší výšce}. Dokud není kletba splněna, hledači nesmí položit \textbf{žádnou otázku}.
	
	\card[K\pikyb]{Kletba nohou hazardního hráče}[1] Během následujících \timecursegambler/, kdykoliv chtějí hledači jít jakýmkoliv směrem, musí hodit 1d8. Následně mohou udělat pouze tolik kroků, kolik padlo. Pak mohou házet znovu.
	
	\cost Musíš uspět hod \textbf{1d20, DC 11}, jinak nemá tato kletba žádný efekt.
	
	\card[A\pikyb]{Kletba černé díry}[1] Urči poloměr kruhu se středem v aktuální pozici hledačů. Tento poloměr musí být menší či roven \textbf{třem čtvrtinám jejich vzdálenosti od tebe}. Po zbytek kola platí, že kdykoliv jsou hledači uvnitř určeného kruhu, \textbf{nesmí pokládat otázky}.
	
	\cost odhoď kletbu z ruky.
	
	\begin{reasoning}
		V původní hře byl poloměr pevně daný hodnotou 75mi, s podmínkou vzdálenosti alespoň 100mi mezi hledači a zónou skrývání. Přepočtená vzdálenost \dist100mi je však příliš velká a nemohl jsem se rozhodnout, jaká by byla adekvátní. Tak mě napadlo to udělat variabilní - čím silnější efekt skrývač použije, tím více informací o své poloze prozradí. Za tuto variabilitu však musí zaplatit další kletbou.
	\end{reasoning}
	
	\card[J\krizeb]{Kletba obratu}[1] Zahraj, pokud hledači právě cestují dopravním prostředkem, jehož následující stanice je od tebe \textbf{vzdálenější} než předešlá. Hledači musí na následující stanici \textbf{vystoupit} a poté, je-li to možné, se vrátit \textbf{stejnou linkou v opačném směru} o alespoň jednu zastávku.
	
	\card[Q\krizeb]{Kletba všímavého spotřebitele}[1] Hledači musí zakoupit produkt či získat vstup na lokaci, na niž viděli reklamu. Tato reklama musí být nalezena v \textbf{reálném světě} (nikoliv na digitálním zařízení hledače) a musí být od samotného produktu či lokace vzdálena alespoň 100m. Dokud není kletba splněna, hledači nesmí položit \textbf{žádnou otázku}.
	
	\cost následující otázka hledačů je zdarma.
	
	\begin{reasoning}
		Protože hrajeme v Praze, je tato kletba spíše lehká. Tak jsem se rozhodl alespoň trochu zvýšit limit vzdálenosti.
	\end{reasoning}
	Otázka: počítá se trh? Je výloha reklama? Je menu v restauraci reklama? A co plakát před fastfoodem?
	\card[K\krizeb]{Kletba zatáčky vpravo}[1] Během následujících \timecurseturnright/ musí hledači na každé křižovatce (včetně chodeb v budovách apod.) \textbf{odbočit vpravo} (o libovolný úhel $\phi \in (0, \pi)$), je-li to možné. Není-li to možné, musí odbočit \textbf{o co nejmenší úhel doleva}. Pokud se hledači ocitnou ve slepé uličce či v situaci, že by museli jít dále než \dist.5mi bez jediné křižovatky s odbočkou vpravo, mohou se otočit o $\pi$.
	
	\cost odhoď 1 kletbu a 1 další libovolnou kartu z ruky.
	
	\card[A\krizeb]{Kletba vymytého mozku}[1] Vyber tři otázky různých typů. Hledači nesmí do konce kola tyto otázky položit.
	
	\cost odhoď všechny karty z ruky.

	\card[J\srdcer]{Kletba nezkušeného cestovatele}[1] Hraj, pokud hledači právě \textbf{necestují} dopravním prostředkem. Urči libovolnou veřejně dostupnou lokaci vzdálenou nejvýše \dist.5mi od hledačů. Ti tam musí dojít a strávit tam alespoň \timecursetravelvisit/. \textbf{Až poté mohou položit další otázku.}

	Hledači ti musí z dané lokace přinést nějaký objekt jako \textbf{suvenýr}. Ztratí-li ho předtím, než ti ho předají, je ti udělen časový bonus \timecursetravelpenalty/.

	\cost cílová lokace od tebe musí být dál než aktuální pozice hledačů.
	
	\card[Q\srdcer]{Kletba zoologa}[1] Vyfoť divokou rybu, ptáka, savce, plaze, obojživelníka či hmyz. Hledači musí vyfotit divoké zvíře stejné kategorie. Do té doby nesmí položit \textbf{žádnou otázku}.
	
	\card[K\srdcer]{Kletba společenstva Dveří}[1] Hraj, pokud hledači \textbf{nejsou v zóně skrývání}. Hledači musí projít \textbf{37 různými dveřmi}, a to takovými, které nelze otevřít či zavřít jinak, než \textbf{ručně}. Dokud není kletba splněna, nesmí hledači položit \textbf{žádnou otázku}.
	
	\cost odhoď z ruky časové bonusy v souhrnné délce alespoň \textbf{\timecursefellowshipcost/}.

    Otázka: já chápu že 37 je skvělé číslo ale to sakra hodně dveří ��. Jenom se bojím aby to nebylo moc.

	
	\card[J\karyr]{Kletba olympijské vesnice}[1] Hledači musí najít alespoň 5 různých vlajek nějakých geografických oblastí. Nemusí být nutně schopni je identifikovat. Tyto vlajky se musí nacházet v okolním světě, nikoliv na objektech ve vlastnictví hledačů. Dokud není kletba splněna, nesmí hledači položit \textbf{žádnou otázku}.
	
	\cost odhoď jednu kartu z ruky.
	
	\card[Q\karyr]{Kletba ptačího průvodce}[1] Když zahraješ tuto kartu, máš jeden pokus natočit co nejdelší \textbf{nepřetržitý záběr ptáka}. Točíš-li ptáků více zároveň, musíš se předem rozhodnout, \textbf{kterého} z nich sleduješ. Záběr končí, jakmile daný pták není viditelný ani na jediném pixelu, nebo jakmile jeho délka přesáhne \textbf{\timecursebirdguide/}. Hledači poté musí natočit záběr ptáka o stejné nebo větší délce. Dokud není kletba splněna, hledači nesmí položit \textbf{žádnou otázku}.

	\card[K\karyr]{Kletba operátora dopravního podniku}[1] Hraj, pokud hledači \textbf{nejsou v zóně skrývání}. Hledači musí najít zastávku, kde jezdí alespoň tři různé linky povolených dopravních prostředků, a správně tipnout, jaká přijede první. V případě neúspěchu mohou tip libovolně mnohokrát opakovat. Dokud není kletba splněna, nesmí hledači používat jízdní řády v jakékoliv formě.

	\cost dokud držíš tuto kartu v ruce, nesmíš používat jízdní řády v jakékoliv formě.
	Otázka: nechápu v čem spočívá nevýhoda že skrývač nemůže používat jízdní řády

	\card[J\pikyr]{Kletba Honzy Žižky}[1] Hraj, pokud je vzdálenost hledačů od Žižkovské věže nejvýše dvojnásobek jejich vzdálenosti k tobě. Hledači musí vyfotit Žižkovskou věž tak, aby byla vidět všechna její patra. Dokud není kletba splněna, hledači nesmí položit \textbf{žádnou otázku}.
	
	\card[Q\pikyr]{Kletba ukrytého oběšence}[1] Hledači musí porazit skrývače ve hře \textbf{šibenice}, jež začíná okamžitě.
	
	Skrývač zvolí \textbf{libovolné spisovné české slovo} mající alespoň 4 písmena, hledači následně písmena hádají. Po \textbf{sedmi} neúspěšných pokusech prohrávají. Hraje se se standardní českou abecedou vyjma diakritiky, tj. \textbf{28 písmen}.
	
	Skrývač musí na každý dotaz reagovat do 30 vteřin. Po porážce nemůžou hledači v následujících \timecursehangmancooldown/ách skrývače vyzvat k další hře. Po třech porážkách či jedné výhře je kletba zrušena.
	
	Dokud není kletba zrušena, hledači nesmí položit \textbf{žádnou otázku} ani nastoupit do \textbf{žádného dopravního prostředku}.
	
	\cost odhoď 2 karty z ruky.

	\card[K\pikyr]{Kletba návštěvníka z dalekých krajů}[1] Najdi nějakou státní poznávací značku, u níž jsi schopen určit oblast původu. Hledači pak musí najít státní poznávací značku pocházející z oblasti, která je stejně či více vzdálená. Dokud není kletba splněna, nesmí hledači položit \textbf{žádnou otázku}.
	
	\card[A\pikyr]{Kletba zabouchnutých dveří}[1] Po následující \timecursedoorduration/ platí, že kdykoliv chtějí hledači projít dveřmi do/z budovy či dopravního prostředku, musí uspět hod \textbf{1d20, DC 9}. Neuspějí-li, nesmí do/z daného prostoru projít, a to \textbf{ani jinými dveřmi}. Po uplynutí \timecursedoorreattempt/ se mohou pokusit uspět znovu.
	
	\cost odhoď libovolnou kartu z ruky.
	Otázka: jenom pro kontrolu, když jedou vlakem a hodí 8 tak musí jet tím samým vlakem 10 minut? Mohou vyjet takto z hrací zóny?

	\card[J\krizer]{Kletba luxusního auta}[1] Vyfoť libovolné auto a zašli onu fotku hledačům. Hledači pak musí vyfotit auto o stejné nebo vyšší pořizovací ceně. Do té doby nesmí položit \textbf{žádnou otázku}.
	
	\card[Q\krizer]{Kletba ztraceného turisty}[1] Pošli hledačům nepřiblížený obrázek ze streetview libovolné mapové služby. Snímek musí být rovnoběžný s horizontem a musí obsahovat alespoň jednu lidmi vytvořenou strukturu jinou než silnici či cestu. Daná lokace musí být vzdálena nejvýše \dist.25mi od současné pozice hledačů.
	
	Hledači musí bez použití digitální technologie dojít na dané místo a poslat ti fotku pro ověření. Dokud není kletba splněna, hledači nesmí položit \textbf{žádnou otázku} ani použít \textbf{dopravní prostředek}.
	
	\cost Hledači se musí nacházet venku.

Poznámka: dát cenu za seslání jako podmínku na začátek jako to je u ostatních karet.

	\card[K\krizer]{Kletba mostního trolla}[1] Zahraj, pokud jsou od tebe hledači vzdáleni nejméně \dist50mi. Hledači musí položit svoji následující otázku ve chvíli, kdy jsou \textbf{pod mostem}.

	\card[A\krizer]{Kletba vybíravého cestovatele}[1]
	\begin{itemize}
		\item najdi tramvaj, hledači po následující hodinu mohou používat jen linky obsahující číslo stejné parity
		\item vyfoť tramvaj a pošli hledačům, po následující hodinu je jediný dopravní prostředek, který mohou hledači používat, tramvaj stejného typu
	\end{itemize}
	
	\card[\cross]{Curse of the ransom note}[0]
	
	\card[\cross]{Kletba citronového amuletu}[0] Každý hledač musí najít citrón a připevnit ho na vnější vrstvu oblečení či kůži. Do té doby nesmí položit žádnou otázku. Pokud v celém zbytku kola nastane situace, že se některý z těchto citrónů nedotýká hledače, je ti udělen časový bonus \timecurselemonpenalty/.
	
	\cost odhoď nějakou akční kartu z ruky.
\end{cards}
